\svnInfo $Id$

%% ==============================================================================
%%
%% Table with the attributes attached to the Spectral Coordinate group.
%%
%% ==============================================================================

\begin{table}[htbp]
  \centering
  \begin{tabular}{|lllp{5.5cm}|} 
    \hline
    \textsc{Field/Keyword} & \textsc{Type} & \textsc{Value} & \textsc{Description} \\
    \hline \hline
    \small \verb|GROUPTYPE| & \small\verb|string| & \small \verb|`SpectralCoord'| &
    Group type descriptor \\
    \small \verb|COORDINATE_TYPE| & \small\verb|string| & \small \verb|`Spectral'| &
    Coordinate Type descriptor \\
    \small \verb|STORAGE_TYPE| & \small\verb|array<string,1>| & \small
    \verb|`Linear'| $|$ \verb|`Tabular'|
    & Descriptor for the underlying storage type for this coordinate \\
    \small \verb|REFERENCE_FRAME| & \small\verb|string| &  & Reference
    position w.r.t. which the spectral coordinate axis are defined;
    see Tab. \ref{tab:reference frames location} for a list of recognized
    values. This can be a diferent frame as used for e.g. the
    direction coordinate or as noted in the coordinates group. \\
    \small \verb|REST_FREQUENCY| & \small\verb|double| &  & Rest
    frequency, $\nu_0$ \\
    \small \verb|REST_FREQUENCY_UNIT| & \small\verb|string| & \verb|`Hz'| & Physical
    units within which the rest frequency is given \\
    \small \verb|REST_WAVELENGTH| & \small\verb|double| &  & Rest
    wavelength, $\lambda_0$ \\
    \small \verb|REST_WAVELENGTH_UNIT| & \small\verb|string| & \verb|`m'| & Physical
    units within which the rest wavelength is given \\
    \small\verb|NOF_AXES| & \small\verb|int| & $N \equiv 1$ & Number
    of coordinate axes \\
    \small \verb|AXIS_NAMES| & \small\verb|array<string,1>| &
    $[name_{0}]$ & World axis names \\
    \small \verb|AXIS_UNITS| & \small\verb|array<string,1>| &
    $[unit_{0}]$ & Physical units along each coordinate axis. \\
    \hline \hline
    \small \verb|REFERENCE_VALUE| & \small\verb|array<double,1>| &
    $[val_{0}]$ & Coordinate value at the reference point \\
    \small\verb|REFERENCE_PIXEL| & \small\verb|array<double,1>| &
    $[pix_{0}]$ & Array location of the reference point in pixels. \\
    \small\verb|INCREMENT| & \small\verb|array<double,1>| &
    $[incr_{0}]$ & Coordinate increment at reference point. \\
    \small\verb|PC| & \small\verb|array<double,1>| &
    $[p_{00}] \equiv 1$ & Non-singular square
    matrix, for the transformation from intermediate pixel coordinates
    to intermediate world coordinates. \\
    \hline \hline
    \small{\verb|AXIS_LENGTH|} & \small\verb|int| & $N_{\rm Pixels}$ &
    Length of the axis, i.e. the number of elements stored in the
    \verb|AXIS_VALUES_PIXEL| and \verb|AXIS_VALUES_WORLD| arrays. \\
    \small{\verb|AXIS_VALUES_PIXEL|} & \small\verb|array<double,1>| &
    $[p_{0},..,p_{N_{\rm Pixels}}]$ & Tabulated values along the pixel
    axis. \\
    \small{\verb|AXIS_VALUES_WORLD|} & \small\verb|array<double,1>| &
    $[w_{0},..,w_{N_{\rm Pixels}}]$ & Tabulated values along the world
    axis. \\
    \hline
  \end{tabular}
  \caption[Keywords decribing a Spectral Coordinate]{Keywords decribing a Spectral Coordinate; attributes within
  the first segment of the table will be present independent of the
  specific storage method.}
  \label{tab:coord-spectral}
\end{table}
