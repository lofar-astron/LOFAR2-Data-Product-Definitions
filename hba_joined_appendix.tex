\svnInfo $Id$

The following text and email exchange summarizes topic of the HBA\_JOINED antenna field discussion.  This issue will be picked up in 2012 for ICD003 and ICD006 for the possibility of adding additional information to the ICDs in order to calculate the beam model per observation.

A2/JMG/AR discussed HBA\_JOINED as new Antenna field;  the Pulsar Working Group would like this info to create beam model in ICD003.  What is needed is:
\begin{itemize}
\item Health status of antennas during the observation, calibration, ...
\item HBA\_JOINED: antenna subfields have different orientation with  
respect to north! This breaks a number of assumptions on antennafield
 homogeneity
\end{itemize}

A2 emailed Ger asking for details on where this information is  
stored. (see thread below).

The meeting notes are as follows:

\begin{verbatim}

-------------------------------------------------------------------------
Data Formats Splinter Meeting (HBA joined), 2011-09-02
-------------------------------------------------------------------------

1 Setting of the meeting
=========================

Present via Skype: Anastasia Alexov (A2), Jean-Mathias Griessmeier (JMG),
Adriaan Renting (AR)  

Minutes: JMG

2 Notes from the meeting
=========================

ICD-003 and ICD-006 have no detailed description of antenna field,
whereas the MS description does.

But calibration (esp. polarization calibration) requires the beamshape!
However, the calibration may be done before the data are written to disk.

For comparison, MS description has a 3 table: 
- stations 	CS001
- antennas 	CS001_HBA0, CS001_HBA1, CS001_HBA
- antenna field CS001_HBA0, CS001_HBA1 (only of antenna is CS001_HBA)
  incl. antenna field orientation per antenna field, ...

See ms2_description_for_lofar_2.07.01.pdf for details.

This is how the system works:
BG outputs contains polarization, but not yet the polarization
calibration. This is currently done offline, but it may be moved to BG.
If this happens, beamshapes etc. are not required in ICD_003/006.
Depending on whether this implementation changes, the tables from the MS
description  will be added to ICD_003/006... or not.

Health status will be added to SYS_LOG.

TODO:
- A2 - discuss this with Aris N.


(email thread)

Hi Ger,

There are a couple of questions which have arisen in the ICD group, 
which we're hoping you could shed some light on or point me to someone 
who could answer these.  We want to know the source of some of the 
MeasurementSet metadata information.  Some of the same information 
will be needed in other LOFAR data products for non-interferometric 
observations.

1) Some of the ICDs will require tracking information of the 
pointings, at some regular time internal, to be stored in tabular 
form.  This is similar to the "field" and "pointing" table in the 
MeasurementSet.  Where is the pointing information stored within the 
system (say for cases when it's not written to a MS file)?  Is it 
stored at a particular time interval already per OBSID?  How could one 
best retrieve this information in order to pack it into the output 
product other than a MeasurementSet?

2) There are several Antenna field-related tables in the 
MeasurementSet, such as the HBA_JOINED table.  These contain the 
health and status of the antennas observation for calibration.  Some 
of the other data formats (Dynamic Spectra and Beam-Formed) need 
similar information about the health and status of the antennas per 
OBSID.  Where can we pull this information from the LOFAR system?  Is 
is available at certain time intervals or just start/end of 
observations?

Thanks much in advance.

best,
A2

Hi Anastasia,

The pointing table is not filled in for the LOFAR MS.
The only pointing info filled is in the FIELD table which 
contains (in J2000) the DELAY_DIR and PHASE_DIR, thus the 
delay center and phase center. Usually they are the same, 
but not necessarily. The info is coming from the parset 
file and written by Chris' data writer.
Calculations of AzEl or apparent coordinates are done on 
the fly (using casacore's Measures classes). It is possible 
to add a virtual column to the MS main table giving AzEL 
or HA. There are also TaQL functions doing it.

The info needed for the beam calibration is filled in by the 
data writer using the BeamTables class (in LOFAR/LCS/MSLofar).
It gets its info from several sources:
- The station config files (text files) tell the layout of a 
station (one file per station). Such a file is read by the 
AntField class.
- The iHBADeltas.conf text files tell the dipole 
	layout of HBA tiles for each station.
- The AntennaSets text file defines for each configuration 
	(like LBA_INNER) which tiles/dipoles are used.
- The SAS DB contains info about broken dipoles/tiles. Ruud 
	has written a query to extract that info and Sven is working
	on a program addbeaminfo to store that info in the MS.

I hope this helps.
I assume you can see our source files, but I can send some if you want to.

Cheers,
Ger

\end{verbatim}
