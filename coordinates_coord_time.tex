\svnInfo $Id$

%%
%%  Section: Time Coordinate
%%

Given the characteristics of the time axis, a time coordinate internal
will either be storing its values as a linear axis
(\verb|STORAGE_TYPE=`Linear'|) or as a 1-dimensional look-up table
(\verb|STORAGE_TYPE=`Tabular'|).

\input coordinates_coord_time_table

\begin{list}{\textbf{--}}{}
\item \verb|GROUPTYPE| is the group type descriptor with the fixed
  value \verb|`TimeCoord'|.
\item \verb|COORDINATE_TYPE| is the coordinate type descriptor with
  the fixed value \verb|`Time'|.
\item \verb|STORAGE_TYPE| indicates the underlying storage mechanism: if 
  \verb|STORAGE_TYPE=`Linear'| the coordinate axis is expected to be linear and
  represented by the attributes defined for a Linear Coordinate (see section
  \ref{sec:coord-linear}). If set \verb|STORAGE_TYPE=`Tabular'|, the values along
  the coordinate axis are expected to be tabulated, thereby represented by the
  attributes defined for a Tabular Coordinate (see section
  \ref{sec:coord-tabular}).
  \begin{comment}
    Add description of structure depending on storage type.
  \end{comment}
\item \verb|REFERENCE_FRAME| records the reference frame within which
  the time coordinate axis is defined; see
  Tab.~\ref{tab:reference-frames-time} for a list of recognized
  values. This can be a diferent frame as used for e.g. the direction
  coordinate or as noted in the coordinates group. 
\item \verb|NOF_AXES| is the number of coordinate axes.
\item \verb|AXIS_NAMES| are the world axis names connected with the
  coordinate axes, i.e. \verb|AXIS_NAMES=[`Time']|.
\item \verb|AXIS_UNITS| are the physical units world axis of the coordinate
  (corresponding to the FITS keyword \verb|CUNITi|, see \cite{fits.paper1}).
  Restrictions on the nature and range of units, if any, will be determined by
  agreements applying to the specific axis. If they are not so limited, units
  should conform to the IAU Style Manual \cite{mcnally.1988}.
\end{list}
