\svnInfo $Id$

%%
%%  Section: Tabular Coordinate
%%

\begin{comment}
  Add paragraph outlining motivation and need for tabular coordinate:
  representation of non-linear (tabulated) 1-dimensional coordinate
  axis.
\end{comment}

\input coordinates_coord_tabular_table

\begin{lstlisting}[float,caption={Structure of the tabular coordinate group.}]
.
`- COORDINATE_{N}
   |- GROUPTYPE             Group       string
   |- COORDINATE_TYPE       Attr.       string
   |- STORAGE_TYPE          Attr.       string
   |- NOF_AXES              Attr.       int
   |- AXIS_NAMES            Attr.       array<string,1>
   |- AXIS_UNITS            Attr.       array<string,1>
   |- AXIS_LENGTH           Attr.       int
   |- AXIS_VALUES_PIXEL     Attr.       array<double,1>
   `- AXIS_VALUES_WORLD     Attr.       array<double,1>
\end{lstlisting}

\begin{list}{\textbf{--}}{}
\item \verb|GROUPTYPE| is the group type descriptor with the fixed value 
  \verb|`TabularCoord'|. 
\item \verb|COORDINATE_TYPE| is the is the descriptor for the
  coordinate type, of value \verb|`Tabular'|.
\item \verb|STORAGE_TYPE| is the descriptor for the underlying storage
  type for this coordinate, of value \verb|`Tabular'|.
\item \verb|NOF_AXES| is the number of coordinate axes; keep in mind that a
  coordinate can consist of multiple axes.
\item \verb|AXIS_NAMES| are the world axis names connected with the
  coordinate axes, e.g.
  \begin{verse}
    \verb|AXIS_NAME=[`Distance']| \\
    \verb|AXIS_NAME=[`Time']|
  \end{verse}
\item \verb|AXIS_UNITS| are the physical units along each coordinate axis
  (corresponding to the FITS keyword \verb|CUNITi|, see \cite{fits.paper1}).
  Restrictions on the nature and range of units, if any, will be determined by
  agreements applying to the specific axis. If they are not so limited, units
  should conform to the IAU Style Manual \cite{mcnally.1988}.
\item \verb|AXIS_VALUES_PIXEL| are the tabulated values of pixel coordinates.
\item \verb|AXIS_VALUES_WORLD| are the tabulated values of world coordinates.
\end{list}
