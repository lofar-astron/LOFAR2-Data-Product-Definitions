\svnInfo $Id$

%%
%%  Section: Polarization Coordinate
%%

\input coordinates_coord_polarization_table

\begin{list}{\textbf{--}}{}
\item \verb|GROUPTYPE| is the group type descriptor with the fixed
  value \verb|`PolarizationCoord'|.
\item \verb|COORDINATE_TYPE| is the is the descriptor for the
  coordinate type, of value \verb|`Polarization'|.
\item \verb|STORAGE_TYPE| is the descriptor for the underlying storage
  type for this coordinate, of value \verb|`Tabular'|.
\item \verb|NOF_AXES| is the number of coordinate axes represented by
  this coordinate; as the Polarization coordinate consists of a single
  tabulared axis, we have \verb|NOF_AXES = 1|.
\item \verb|AXIS_NAMES| are the world axis names connected with the
  coordinate axes; for a Polarization coordinate \verb|AXIS_NAMES = `Polarization'|.
\item \verb|AXIS_UNITS| are the physical units along each coordinate axis
  (corresponding to the FITS keyword \verb|CUNITi|, see \cite{fits.paper1}).
  Restrictions on the nature and range of units, if any, will be determined by
  agreements applying to the specific axis. If they are not so limited, units
  should conform to the IAU Style Manual \cite{mcnally.1988}. \\ The
  units of the Stokes parameters I, Q, U and V, of total polarization
  (linear, elliptical or circular) and of separate circular
  polarizations (L, R) are some form of flux density. 
\item \verb|AXIS_VALUES_PIXEL| holds the tabulated values along the pixel axis
\item \verb|AXIS_VALUES_WORLD| holds the tabulated values along the world
  axis of the Polarization coordinate, i.e. the names of the Polarization
  components. Commonly used values are:
  \begin{center}
    \begin{tabular}{lp{10cm}}
      \verb|AXIS_VALUES_WORLD| & \textsc{Description} \\
      \hline
      \verb|[`I']| & Total flux density only data. \\
      \verb|[`I',`Q',`U',`V']| & Full set of standard Stokes
      parameters. \\
      \verb|[`X',`Y']| & Raw time-series TBB data, originating
      directly from the individual dipoles. \\
      \verb|[`XX',`YY',`XY',`YX']| & Cross-correlation products from a
      pair of $X$-linear and $Y$-linear receiver feeds. \\
      \verb|[`R',`L',`X',`Y']| & $X$/$Y$ linear components, as well as
      $R$/$L$ circular components. \\
      \hline
    \end{tabular}
  \end{center}
  
 For a full list of recognized values and their description see
  Tab. \ref{tab:polarization-values} below.
  \begin{table}[ht]
    \centering
    \begin{tabular}{|llp{10cm}|}
      \hline
      \textsc{Term} & \textsc{Symbol} & \textsc{Description} \\
      \hline \hline
      Stokes Parameters & I & Standard Stokes total intensity,
      i.e. total Poynting vector or flux density of the wave. \\
      & Q  & Standard Stokes linear; degree of polarization, i.e. the
      difference in intensities between horizontal and vertical
      linearly polarized components. \\
      & U  & Standard Stokes linear; plane of polarization, i.e. the
      difference in intensities between linearly polarized components
      oriented at $\pm \pi/4$ w.r.t. the components of $Q$ \\
      & V  & Standard Stokes circular; ellipticity, i.e. the differences
      in intensities between right and left circular polarized
      components. \\
      \hline
      Circular feeds & R  & Right circular \\
      & L  & Left circular \\
      & RR & Right-right circular \\
      & LL & Left-left circular \\
      & RL & Right-left circular \\
      & LR & Left-right circular \\
      \hline
      Linear feeds & X  & X linear \\
      & Y  & Y linear \\
      & XX & X parallel linear \\
      & YY & Y parallel linear \\
      & XY & XY cross linear \\
      & YX & YX cross linear \\
      \hline
    \end{tabular}
    \caption{Recognized values for the Polarization component parameter.}
    \label{tab:polarization-values}
  \end{table}
\end{list}
