\svnInfo $Id$

%%
%%  Section: Direction Coordinate
%%

The Direction Coordinate consists of a set of two coupled coordinate
axes, describing a direction in space; it therefore includes
information such as the equinox of the observation, the system of
equatorial coordinates on the sphere of the sky, as well as parameters
for the spherical map projection. The attributes storing the actual
coordinate parameters are listed in Tab.~\ref{tab:coord-direction} below.

\input coordinates_coord_direction_table

Several systems of equatorial coordinates (right ascension and
declination) are in common use. Apart from the International Celestial
Reference System (ICRS, IAU, 1984), the axes of which are by
definition fixed with respect to the celestial sphere, each system is
parameterized by time. In particular, mean equatorial coordinates are
defined in terms of the epoch (i.e.\ instant of time) of the mean
equator and equinox (i.e.\ pole and origin of right ascension). The
same applies for ecliptic coordinate systems. The keyword
\verb|RADEC_SYS| is used to specify the particular system; recognized
values are given in Tab. \ref{tab:radec-sys} below.

\begin{table}[ht]
  \centering
  \begin{tabular}{ll}
    \hline
    \verb|RADEC_SYS| & \textsc{Description} \\
    \hline \hline
    \verb|ICRS|     & International Celestial Reference System \\
    \verb|FK5|      & mean place, new (IAU 1984) system \\
    \verb|FK4|      & mean place, old (Bessell-Newcomb) system \\
    \verb|FK4-NO-E| & meanplace, old system but without e-terms \\
    \verb|GAPPT|    & Geocentric Apparent Place, IAU 1984 system \\
    \hline
  \end{tabular}
  \caption{Allowed values of \texttt{RADEC$_-$SYS}}
  \label{tab:radec-sys}
\end{table}
