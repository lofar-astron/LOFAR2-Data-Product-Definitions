\svnInfo $Id$

\begin{description}
\item[Az] Azimuth.
\item[AIPS++] The AIPS++ project was a project from the nineties supposed to replace the original Astronomical Information Processing System or classical AIPS. The ++ comes from it being mainly developed in C++. It's also known as AIPS 2. It evolved into CASA, casacore and casarest (see those entries).
\item[BBS] BlackBoard Selfcal, pipeline used for LOFAR imaging data.
\item[Beam] A beam is formed by combining all the SubArrayPointing,
one for each station, which are looking in a particular direction.
There may be more than one beam for each SubArrayPointing, and different types
of beams are available.
\item[BF] Beam-Formed data (time series structure).
\item[CASA] The Common Astronomy Software Applications package. User software for radioastronomy developed out of the old AIPS++ project. The project is led by NRAO with contributions from ESO, CSIRO/ATNF, NAOJ and ASTRON. \cite{website:casa}
\item[casacore] The set of C++ libraries that form the basis of CASA and several other astronomical packages. It contains classes for storing and handling visibility and image data, RDBMS-like table system and handling coordinates. Mainly maintained by ASTRON and CSIRO/ATNF. \cite{website:casacore}
\item[casarest] The libraries and tools from the old AIPS++ project that are not part of casacore or CASA but still in use.
\item[CEP] Central Processing facility.
\item[Channel] The subband data of a LOFAR observation may be passed through a
second polyphase filter to obtain a large number of channels (i.e. to increase
the spectral resolution).
\item[CLA] Common LOFAR attributes. Set of root-level attributes that
  are used and required as attributes in all LOFAR science data
  products. If a value is not available for an Attribute,
  \verb|`NULL'| maybe used.
\item[Co-I] Co-investigators on an observation project under the leadership of
the PI.
\item[Data Interface] Set of definitions that describe the contents
  and structure of data files.
\item[Data Access Layer (DAL)] A \cpp\ library with Python bindings
  providing read/write functionality for HDF5 format files, as well as
  access to Measurement Sets.
\item[Dec] Declination.
\item[DPPP] Default Pre-Processing Pipeline, pipeline used for LOFAR
  imaging data.
\item[EAS] Extensive Air-Shower.
\item[El] Elevation.
\item[FITS] FITS (Flexible Image Transport System) is a digital file format
  used to store, transmit, and manipulate scientific and other images. FITS
  commonly used in astronomy.
\item[HBA] High Band Antenna.
\item[HDFView] Hierarchical Data Format Viewer;  a Java software tool
  for viewing the HDF5 structure and
  data. [http://www.hdfgroup.org/hdf-java-html/hdfview/]
\item[HDF5] Hierarchical Data Format, 5 \cite{hdf5}. A file format capable of
  accommodating large datasets that comprises two (2) primary types of
  objects: groups and datasets. Implements self-organisation and
  hierarchical structures within the file format itself, facilitating
  self-contained data administration. \cite{hdf5.rm,hdf5.ug}
\item[HDF5 group] A grouping structure containing zero or more HDF5
  objects, together with supporting meta-data.
\item[HDF5 dataset] A multidimensional array of data elements,
  together with supporting meta-data.
\item[HDU Header-Data Unit] Though typically used for FITS data
  descriptions, the term ``HDU'' can also be used more generically when
  discussing any data group that contains both data and a descriptive
  header.
\item[Hypercube] The hypercube is a generalization of a 3-cube to $n$
  dimensions, also called an $n$-cube or measure polytope. In data
  modelling a hypercube is a cube-like logical model in which all
  measurements are organized into a multidimensional space.
\item[ICD] Interface Control Document.
\item[IVOA] International Virtual Observatory Alliance.
\item[KSP] Key Science Project.  One of several major observational
  and research projects defined by the LOFAR organization. These Key
  Science Projects are,
  \begin{itemize}
  \item Cosmic Magnetism in the Nearby Universe
  \item High Energy Cosmic Rays
  \item Epoch of Re-ionization
  \item Extragalactic Sky Surveys
  \item Transients - Pulsars, Jet Sources, Planets, Flare stars
  \item Solar Physics and Space Weather
  \end{itemize}
\item[LBA] Low Band Antenna.
\item[LOFAR] The LOw Frequency ARray. LOFAR is a multipurpose sensor array;
  its main application is astronomy at low radio frequencies, but it also has
  geophysical and agricultural applications.  [http://www.lofar.org/]
\item[LOFAR Sky Image] Standard LOFAR Image Cube.  A LOFAR data
  product encompassing science data, associated meta-data, and
  associated calibration information, including a Local Sky Model
  (LSM) , and other ancillary meta groups that are defined in this
  document.
\item[LSM/GSM] The Local Sky Model/Global Sky Model.  Sky Models are
  essentially catalogues of known real radio sources in the sky.  A
  Local Sky Model for an observation is merely a subset of a Global
  Sky Model catalogue pertaining to that observation's relevant region
  of the sky.
\item[LTA] The Long Term Archive for LOFAR.
\item[MJD] Modified Julian Day. Derived from Julian Date (JD) by MJD =
  JD -  2400000.5. Starts from midnight rather than noon.
\item[MS] Measurement Set, a self-described, structured set of casacore
  tables comprising the data and meta-data of an observation. \cite{aips++.note.229}
\item[PI] A Principal Investigator is the lead scientist resopnsible for a
particular observation project.
\item[RA] Right Ascension.
\item[RFI] Radio Frequency Interference.
\item[RM] Rotation Measure.
\item[RMSC] The Rotation Measure synthesis cube is a data product which contains
the output of LOFAR RM synthesis routines, namely the polarized emission as a
function of Faraday depth.  As with the Sky Image data files, all associated
information is stored within an RMSC file.
\item[RSP] Remote Station Processing Board.
\item[SIP] Standard Imaging Pipeline or Submission Information Package within the context of the LTA.
\item[Station] Group of antennae separated from other groups. In it's current
cofiguration, LOFAR has 48 stations.
\item[SubArrayPointing] This corresponds to the beam formed by the sum of all of
the elements of a station. For any given observation there may be more
than one SubArrayPointing, and they can be pointed at different locations.
\item[Subband] At the station level, LOFAR data are passed through a polyphase
filter, producing subbands of either 156.250 kHz or 195.3125 kHz (depending on
system settings).
\item[TAI] International Atomic Time (Temps Atomique International),
  atomic coordinate time standard.
\item[TBB] Transient Buffer Board.
\item[TRAP] Transients Pipeline.
\item[USG] LOFAR User Software Group.
\item[UTC] Coordinated Universal Time (UTC) is a time standard based
  on International Atomic Time (TAI) with leap seconds added at
  irregular intervals to compensate for the Earth's slowing rotation.
\item[UV-Coverage] A spatial frequency domain area that must be covered
completely by observation in order to assure an optimal target image (Full UV-
Coverage). During observation, the radio telescope turns with respect to its
target, due to the earth rotation. A certain -instrument geometry dependent-
rotation angle has to be covered in order to accomplish full coverage.
\item[VHECR] Very high-energy cosmic ray.
\item[WCS] World Coordinate Information (WCS). The FITS "World
  Coordinate System" (WCS) convention defines keywords and usage that
  provide for the description of astronomical coordinate systems in a
  FITS image header \cite{fits.paper1, fits.paper2, fits.paper3}.
\end{description}
