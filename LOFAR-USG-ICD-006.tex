
\documentclass[a4paper,10pt,bibtotoc]{scrartcl}
\usepackage{a4wide,usg}
\usepackage{draftcopy}

%% Do NOT remove: required to extract SVN information
\svnInfo $Id$

%% Adjust page footer
\fancyfoot[LE,LO]{LOFAR-USG-ICD-006: Dynamic Spectrum Data}
\fancyfoot[RE,RO]{\textsc{lofar} Project}

\begin{document}

%%_______________________________________________________________________________
%% Titlepage

\title{LOFAR Data Format ICD \\ Dynamic Spectrum Data \\
{\normalsize Document ID: LOFAR-USG-ICD-006} \\ 
{\normalsize Version 2.03.10} \\
{\normalsize SVN Repository Revision: \svnInfoMaxRevision}}
\author{J.-M. Grie{\ss}meier, A.~Alexov, K.~Anderson, L.~B\"ahren}
\date{\small{SVN Date: \svnInfoMaxToday}}
\maketitle

\tableofcontents
\listoffigures
\listoftables

\clearpage

%%_______________________________________________________________________________
%% Change record of the document

\section*{Change record}
\addcontentsline{toc}{section}{Change record}

\begin{center}
  %% Table head
  \tablefirsthead{
    \hline
    \sc Version & \sc Date & \sc Sections & \sc Description of changes \\
    \hline
  }
  \tablehead{
    \multicolumn{4}{r}{\small\sl continued from previous page} \\
    \hline
    \sc Version & \sc Date & \sc Sections & \sc Description of changes \\
    \hline
  }
  %% Table tail
  \tabletail{
    \hline
    \multicolumn{4}{r}{\small\sl continued on next page} \\
  }
  \tablelasttail{\hline}
  %% Table contents
  \begin{supertabular}{lllp{10cm}}
    0.1  & 2009-09-28 & all & document creation. \\
    0.2  & 2009-11-02 & all & update file names and format (dynspec). \\
    0.3  & 2009-11-18 & all & adjusted document to standard structure. \\
    0.4  & 2010-02-09 & all & rewrote document (not complete). \\
    0.5  & 2010-02-12 & all & continued rewriting of document. \\
    0.6  & 2010-02-19 & all & first complete version.\\
    0.65 & 2010-03-08 & all & minor updates, different keywords. Repl. Figure 1, 2. \\
    0.70 & 2010-04-19 &   4 & Refactor section 4.1 == 4.1.1, 4.1.2 \\
    0.8  & 2010-06-04 & Appendix & Removed section ``Coordinate group examples''
    from the appendix; detailed description and examples now can be found in
    \texttt{LOFAR-USG-ICD-002} (``Representation of World Coordinates''). \\
    0.9  & 2010-07-02 & all & many minor modification and corrections. \\
    2.00.00  & 2010-07-08 & Cover & Changed `revision` to `version`;  updated 
    this version number to 2.00.00 for LOFAR ICDs 1 through 7 to put them on the same
    version numbering scheme.\\
    2.00.02  & 2010-10-28 & all & minor modification and corrections; change of 
    naming scheme to ``sub-array pointing'' and ``beam''. \\
    2.00.03 & 2010-11-23 & all & replaced ``unit'' by ``value'' where appropriate. \\
    2.01.00 & 2011-03-04 & all & Fixed nomenclature of ATTRIBUTES, GROUPS and
    DATASETS. Expanded glossary.\\ 
    2.01.01 & 2011-03-08 & all & Changed Figures 4 \& 3. Changed DynSpec to DYN\_SPEC.\\ 
    2.01.02 & 2011-03-10 & all & Maintain list of references through Bib\LaTeX\ database. \\
    2.02.00 & 2011-04-19 & \ref{sec:coordinates group} & renamed ``Frequency'' to ``Spectral''.
    Reworked section 4.5 on coordinates.\\ 
    2.03.00 & 2011-04-27 & \ref{sec:dynspec dataset} & Attributes
    describing shape of data array; fixing section name. \\
    2.03.01 & 2011-05-11 & \ref{sec:coordinates group} & Added
    paragraph with notation conventions. \\
    2.03.02 & 2011-05-23 & all & small changes in notation; moved section 6 to
    appendix; inverted sections 3.2 and 3.3 \\
    2.03.03 & 2011-06-29 & all & Matching up group type attributes and
    notation.\\
    2.03.04 & 2011-07-06 & all & Matching up group type attributes and
    notation; consolidation of labels to refer to standard sections
    and tables. \\
    2.03.05 & 2011-10-25 & all & Changed all `float' types to `double';  
    changed all `bool' types to `unsigned int';  changed all `integer' to
    `int'.\\
    2.03.06 & 2012-01-27 & all & Include standard data types decription. 
    Added standard data types description. Reverted `unsigned int' back to
    `bool'. Made optional keywords \textit{italic}. Incorporated a large number
    of comments. Moved section on dataflow to the appendix.\\ 
    2.03.07 & 2012-02-08 & all & Changed ``dyn\_spec'' to ``dynspec''.\\     
    2.03.08 & 2012-03-06 & cover   & Added draftcopy package for background 'draft' text.\\
    2.03.09 & 2012-05-24 & all   & minor changes.\\
    2.03.10 & 2012-06-19 & \ref{sec:root group} & removed ``nof\_stations'' and ``station\_list''
    from additional root attributes: Is already covered by CLA.\\
    
%    removed \texttt{NOF\_STATIONS} and 
%    \texttt{STATIONS\_LIST} for \texttt{SUBARRAY\_POINTING} \\
  \end{supertabular}
\end{center}

\input version_numbering
\input types
\input notation

\section*{Acknowledgements}

We would like to thank all those who contributed to this document by
proof-reading, providing valuable suggestions, corrections, comments, and
advice. In particular, we would like to thank Frank Breitling and Jochen Eisl\"{o}ffel.

\clearpage

%% _______________________________________________________________________________
%% Introduction

\section{Introduction}
\label{sec:introduction}

%% --------------------------
\subsection{Purpose and Scope}
\label{sec:purpose and scope}

This interface control document (ICD) describes the data format for LOFAR dynamic
spectrum data (in it's simplest version, a dynamic spectrum is 
a two-dimensional time-frequency plot). 
It was derived from the ICDs describing Beam-Formed
Data \cite{lofar.icd.003} and Sky Images \cite{lofar.icd.004}.

LOFAR beams can be formed where signals from antennae and/or stations are
coherently added to form sets of smaller beams, each generating a time-series
voltage sum \cite{vos.2004}.  Alternatively, beams can also be incoherently added to
create total intensities. 

In either case, the information found in each beam can
be represented as Dynamic Spectrum.
This document is intended to be the formal interface control agreement between
the LOFAR project, observers/users of LOFAR data products, and the eventual LOFAR
science archive facility. 

%% --------------------------
\subsection{Context and Motivation}
\label{sec:context and motivation}

A LOFAR Dynamic Spectrum Data file will be the data hosting structure for dynamic
spectrum data produced by LOFAR, irrespective of their scientific purpose.
It is one of the tasks of the LOFAR project to define and describe the
structure of the LOFAR Dynamic Spectrum Data file format.

Dynamic spectra will be useful e.g.~for storing and analyzing planetary data, flare stars,
the Sun, terrestrial lightning, and pulsars. The typical application
is to have one or several beams (for example `ON' and `OFF' beams, to compare 
beams on the source and beams pointing just next to the source for comparison) 
within one
sub-array pointing. In this, the sub-array pointing corresponds to the beam formed by the sum of all of
the elements of a station, whereas a beam is formed by combining all the
SubArrayPointing, one for each station.

Each of these beams produces a dataset accompanied
by a number of by-products, such as flagging information, event
tables, etc. In the more traditional approach, where all such products
are stored and managed separately, a large amount of book-keeping is
required to maintain consistency. For the LOFAR project, a Dynamic
Spectrum Data file product will be defined within the context of the
Hierarchical Data Format 5, or HDF5. HDF5 allows for  storage, not
only of the data, but also for the associated and related meta-data
describing the file contents, conditions of observations, etc. As an
``all-in-one'' wrapper, the HDF5 format simplifies the management of
what are expected to be very large datasets that formats such as FITS
cannot pragmatically accommodate.

% There has been much discussion of a putative need for LOFAR file
% headers to adhere to FITS-like header keywords.  Though it is envisioned that the
% LOFAR project will provide observers and other users with FITS format image files
% upon request, it is not entirely necessary that HDF5 header keywords match FITS
% keyword conventions in a LOFAR file itself.  A format conversion layer
% can certainly be developed to provide rigorous transformation of LOFAR headers
% into more restricted FITS header keyword sets.  However, development of such a
% layer would be simplified in the event that LOFAR files make use of
% \textit{de facto} FITS standard keywords as much as possible.


For the purposes of further discussion regarding Dynamic Spectrum Data file 
adherence to FITS
keyword standards, the \textit{ESO Data Interface Control Document} (\textit{see
\S\ References}), has been adopted as the FITS keyword model. 

\input applicable_documents

%% ______________________________________________________________________________
%%                                                              Section: Overview

\section{Overview}
\label{sec:overview}

LOFAR data will be presented in a number of LOFAR data formats, all of
which will provide data arrays of differing dimensions, depending upon the
respective observation.  Dynamic Spectra, Sky Image Cubes, Rotation Measure
Cubes, Near-field cosmic ray images (``CR image'' in Table
\ref{tab:lofar_data_types}), etc., all have different dimensions and coordinate
types. Table  \ref{tab:lofar_data_types} illustrates the various data
array dimensions that LOFAR may produce. 

\input coordinates_image_table

Each data type is described in detail by an appropriate interface control
document.  This document pertains to, and describes only those data conforming
to the LOFAR datatype ``Dynamic Spectrum''.  The dataset
array in a Dynamic Spectrum will be an n-dimensional array. The Dynamic
Spectrum is designed to store the polarized intensity data
produced from the Dynamic Spectrum pipeline.  The nominal dimensionality of a
Dynamic Spectrum \texttt{Data} group's dataset will be 3 (\texttt{N\_AXIS=3}),
wherein the Dynamic Spectrum cube (or cubes) will be defined in (C-type order)
\textit{Polarization}, \textit{time}, \textit{spectral}, as shown in Table
\ref{tab:lofar_data_types}.

This document is structured as follows: Section \ref{sec:organization of the data} will present
a high-level view of the hierarchical  structure of LOFAR data files, file form,
and semantic conventions the interface will adhere to, including a statement of
the primary data product format, HDF5. These conventions will also include names,
meaning, and physical units that may be used to generate and interpret the data
files.  
Section \ref{sec:detailed structure} will present the low-level
specification for the data, including a description of the structure of LOFAR
Dynamic Spectrum Data files, and the various group entities and sub-structures
comprising these files, i.e. LOFAR group types, units, physical quantities.

%% _______________________________________________________________________________
%% Organization of the data

\section{Organization of the data}
\label{sec:organization of the data}

%% --------------------------

\subsection{High level LOFAR Dynamic Spectrum Data file structure}

A LOFAR Dynamic Spectrum Data file will adhere to the following guidelines:\\
A LOFAR Dynamic Spectrum Data file will be defined within the context of the HDF5
file format.  In an effort to minimize the hierarchical depth of the file
structure, a Dynamic Spectrum Data file is designed to be as ``flat'' as possible,
providing access to the necessary data without undue hierarchical tree crawling. 

Therefore, the HDF5 structure of a Dynamic Spectrum Data file will comprise a
primary group, a 
``ROOT group'' in HDF5 nomenclature, which may be considered equivalent to a
primary header/data unit (HDU) of a standard multi-extension FITS file.  This
primary group will consist only of header  keywords (``attributes'' in HDF5
nomenclature) describing general properties of an observation, along with
pointers to contained subgroups.  

\begin{figure}[htbp]
\begin{center}
  \includegraphics[scale=0.55]{figures/DynSpec.eps}
\end{center}
\caption{Dynamic Spectrum Data file structure.}
\label{fig:high-level structure}
\end{figure}

%% --------------------------

\subsection{Overview of Dynamic Spectrum Groups}

The layout of a LOFAR Dynamic Spectrum Data file is shown in Figure 
\ref{fig:high-level structure}.
A LOFAR Dynamic Spectrum Data file will comprise a \texttt{SYS\_LOG Group}, 
just below the ROOT level, which contains logs and parameter files which are 
relevant to the entire file.  Additionally, just below the ROOT level, the
Dynamic Spectrum Data file will
contain an arbitrary, observation-dependent number of \texttt{DYNSPEC Groups}
containing a \texttt{COORDINATES Group}, a \texttt{DATA Group}, 
an \texttt{EVENT Group}, and a
\texttt{PROCESS\_HISTORY Group}. 

The main building blocks of the Dynamic Spectrum Data HDF5 file are:

\begin{enumerate} \parskip 0pt
\item \textbf{File Root-level} (\verb|ROOT|). The ROOT level of the
  file contains the majority of associated meta-data, describing the
  circumstances of the observation. These data attributes include
  observation time (start and end), frequency window (high band
  vs. low band, filters) and other important characteristics of the
  dataset. See Sections \ref{sec:lofar common attributes} and \ref{sec:additional root group attributes} for details.
  
\item \textbf{System Logs Group} (\verb|SYS_LOG|). This is a catch-all
  envelop encapsulating information about all the system-wide steps of
  processing which are relevant to the entire observation, such as
  parameter sets and processing logs. See Sec.~\ref{sec:sylog group} for details.
  
\item \textbf{DYNSPEC Groups}. Each observation dynamic spectrum
is stored as a
separate group within the file, containing its own set of four sub-groups. 
Characteristics about each dynamic spectrum are stored as Attributes in group
headers. 
A LOFAR Dynamic Spectrum Data file may contain numerous dynamic spectra groups.
Possibilites include:
\begin{itemize}
\item one dynamic spectrum containing the data from all stations where different
stations all observe at the same frequencies, 
\item one dynamic spectrum per station when different stations observe at different 
frequencies (in that case, the combined spectrum may be written to the
\texttt{TILED\_DYNSPEC Group}, see below),  
\item one dynamic spectrum per station where different stations observe at the 
same frequencies (e.g.~to monitor RFI), 
\item separate dynamic spectra for ON and OFF beams,
\item dynamic spectra created from the same file, but with
different frequency and time resolution.
\end{itemize}
Each \texttt{DYNSPEC Group} 
will contain its own \texttt{COORDINATES Group} (see below) 
plus one \texttt{Data} group, 
which will in turn contain a dataset as an
ndarray, along with associated attributes. 
See Section \ref{sec:image group} for details.

\item \textbf{EVENT Groups}. In some (if not all) cases, dynamic spectra will be
monitored for events (flares, bursts of emission, ...) during file creation. In
those cases, an \textbf{EVENT Group} will contain a table of events.
See Section \ref{sec:event group} for details.

\item \textbf{Coordinates Groups} (\verb|COORDINATES|). Each
  \texttt{DYNSPEC Group} contains one \texttt{COORDINATES Group},
  which usually contains 3 Coordinate sub-groups (the TIME, SPECTRAL and
  POLARIZATION coordinates). See Section \ref{sec:coordinates group} for details.

\item \textbf{Processing History Groups} (\verb|PROCESS_HISTORY|) can
  be found on the \texttt{DYNSPEC Group} level. 
  This group contains information about all the steps of
  processing, such as parameter sets and processing logs. See
  Section~\ref{sec:processing group} for details.

\item \textbf{Dynamic Spectrum DATA arrays}. 
For each \texttt{DYNSPEC Group}, the dynamic spectra are stored as ndarrays in 
the respective DATA group - it is at this 
4th hierarchical depth that the bulk of the data reside. The data storage options
are still being investigated, in order to determine the maximum efficiency of
data seeks and file I/O.  
See Section \ref{sec:dynspec dataset} for details.

\item \textbf{TILED\_DYNSPEC Group}. 
In the case where different stations observe at different frequencies,
a  \texttt{TILED\_DYNSPEC Group} can be used
to accommodate a composite dynamic spectrum, combining the frequency information of
different \texttt{DYNSPEC groups}.
See Section \ref{sec:tiled dynspec group} for details.

\end{enumerate}

%% --------------------------

\subsection{Hierarchical Structure of the HDF5 file}

The Dynamic Spectrum Data are organized within a hierarchical structure, which
reflects upon the structure in which data are grouped during processing.  This
structure can be represented as an HDF5 hierarchy in the following way:

\begin{lstlisting}
OBS_NUMBER/
OBS_NUMBER/SYS_LOG
OBS_NUMBER/DYNSPEC_000/COORDINATES
OBS_NUMBER/DYNSPEC_000/COORDINATES/TIME_COORD
OBS_NUMBER/DYNSPEC_000/COORDINATES/SPECTRAL_COORD
OBS_NUMBER/DYNSPEC_000/COORDINATES/POLARIZATION_COORD
OBS_NUMBER/DYNSPEC_000/DATA
OBS_NUMBER/DYNSPEC_000/EVENT
OBS_NUMBER/DYNSPEC_000/PROCESS_HISTORY
...
OBS_NUMBER/DYNSPEC_NNN/...
OBS_NUMBER/TILED_DYNSPEC/...
...
\end{lstlisting}

An alternate way to display the above described HDF5 hierarchy is:

\begin{lstlisting}
OBS_NUMBER
|-- SYS_LOG
|-- DYNSPEC_000
|   |-- COORDINATES
|   |   |-- TIME_COORD
|   |   |-- SPECTRAL_COORD
|   |   `-- POLARIZATION_COORD
|   |-- DATA
|   |-- EVENT
|   `-- PROCESS_HISTORY
|
|-- DYNSPEC_NNN
|   |
|   ...
|
`-- TILED_DYNSPEC
|    |
|    ...
'
\end{lstlisting}

%% ------------------------------------------------------------------------------
%% Section: Detailed Data Specification

\section{Detailed Data Specification}
\label{sec:detailed structure}

\input metadata_intro   

%% ------------------------------------------------------------------------------

\subsection{The ROOT Group (\tt{ROOT})}
\label{sec:root group}

The LOFAR file hierarchy begins with the top level \textbf{File Root
  Group} (\verb|ROOT|). This is the file entry point for the data, and
the file node by which navigation of the data is provided.  The
\textbf{File Root Group} will comprise a set of attributes that describe
the underlying file structure, observational metadata, the LOFAR
Dynamic Spectrum Data, as well as providing hooks to all groups attached to
the \textbf{File Root Group}. 

This section will specify two set of attributes that will appear in the \texttt{ROOT Group}: a set of Common LOFAR Attributes (CLA) that will be common to all LOFAR science data products, and a set of attributes that are specific to LOFAR dynamic spectra.  Though these attributes will all appear together in the \texttt{ROOT} attribute set, they are separated in this document in order to demarcate those general LOFAR attributes that are applicable across all data, and those attributes that are dynamic spectra-specific.

In other words,
\begin{verse}
  \verb| ROOT Attributes = |\\
  \verb| Common LOFAR Attributes (CLA) + Additional (dynamic spectrum specific) ROOT Attributes.|
\end{verse}

The Common LOFAR Attributes are the first attributes of any LOFAR \textbf{File Root Group}.

Table \ref{tab:group types} lists the group types used for LOFAR Dynamic Spectrum
Data.

%% --------------------------
 
\subsubsection{Common LOFAR Attributes}
\label{sec:lofar common attributes}

This section will specify a set of ROOT-level attributes that will be common to
all LOFAR science data products.  These ``LOFAR common metadata'' will appear as attributes
at the ROOT level of all LOFAR Dynamic Spectrum Data files, as well as 
\textit{all} other LOFAR products.  These common LOFAR metadata are to
be the first set of attributes of any LOFAR file ROOT group. 

Table~\ref{tab:lofar common metadata} lists the {\cla} (CLA) which can be found 
in LOFAR Dynamic Spectrum Data within the file's ROOT header. 
These Attributes are required to be in the ROOT Group; if a
value is not available for an Attribute, a \verb|`NULL'| maybe used in its place.

For LOFAR Dynamic Spectrum Data, \verb|FILETYPE|=`dynspec'.

\input lofar_common_metadata

\begin{table}[ht]
  \centering
  \begin{tabular}{|lrp{6cm}|} 
    \hline
    \sc \textbf{General LOFAR Group} & \sc \textbf{Value} & \textbf{Description} \\
    \hline
    Root          & \verb|`Root'|    & Top-level LOFAR group type \\
    System Log    & \verb|`SysLog'|  & System log files, parsets \\
    Tiled Dynspec & \verb|`TiledDynspec'|& Tiled Dynamic Spectrum\\
    Dynspec & \verb|`Dynspec'|   & Dynamic spectrum \\
    \hline \hline
    \sc \textbf{Dynspec Subgroups} & \textbf{Value} & \textbf{Description} \\
    \hline
    Data group  & \verb|`Data'| & This is a Data group of a Dynamic Spectrum
    Data file \\
    Event group & \verb|`Event'| & This is a Event List group \\
    Processing History group &\verb|`ProcessHistory'| & This is a Processing History group \\
    Coordinates Group & \verb|`Coordinates'| & This is a Coordinates group \\
    \hline
    \sc \textbf{Coordinates Group Subgroups} & \textbf{Value} & \textbf{Description} \\
    \hline
    Time coord group      &\verb|`TimeCoord'|      & This is a coord group describing the time axis\\
    Spectral coord group    &\verb|`SpectralCoord'|    & This is a coord group describing the
    frequency axis\\
    Polarization coord group     &\verb|`PolarizationCoord'|  & This is a Stokes coordinate group \\
    \hline
  \end{tabular}
  \caption{LOFAR Dynamic Spectrum Data Group Types.}
  \label{tab:group types}
\end{table}

%% --------------------------

\subsubsection{Additional dynamic spectrum ROOT Attributes}
\label{sec:additional root group attributes}

Table~\ref{tab:root attributes} contains the Dynamic Spectrum Data ROOT header
Attributes.  The CLA Attributes have already been listed in
Section~\ref{sec:lofar common attributes} above, therefore these are additional
attributes in the ROOT header. 
\vspace{20pt}

%\begin{table}[htbp]
%\centering
\tablefirsthead{%
	\hline
	\sc Field/Keyword & \sc H5Type & \sc Type & \sc Unit & \sc Description
  \\
	\hline}
\tablehead{%
	\hline
	\sc Field/Keyword & \sc H5Type & \sc Type & \sc Unit & \sc Description
  \\
	\hline}
\tabletail{%
	\hline
	\multicolumn{5}{|r|}{\small\sl continued on next page}\\
	\hline}
\tablelasttail{\hline}
\bottomcaption{Additional ROOT group attributes for LOFAR Dynamic Spectrum Data.
\label{tab:root attributes}}
\begin{supertabular}{|llllp{5cm}|}
  \hline
%  \sc Field/Keyword & \sc H5Type & \sc Type & \sc Unit & Description \\
  \hline
  \small{\verb|DYNSPEC_GROUPS|}        && bool   &---& 
  	File has \texttt{DYNSPEC} subgroups (value is \verb|`true'|) \\
  \small{\verb|NOF_DYNSPEC|}          && int      &             ---            &
  	number of \texttt{DYNSPEC groups} in this file \\
  \small{\verb|NOF_TILED_DYNSPEC|}          && int      &             ---            & 
  	number of \texttt{TILED\_DYNSPEC groups} in this file (0 or 1)\\  \small{\verb|CREATE_OFFLINE_ONLINE|} & Attr. & \verb|string| & --- & Whether the file was created `Online' (stream to file) or
  `Offline' (file to file). \\
  \small{\verb|BF_FORMAT|} & Attr. & \verb|string| & --- & ``RAW'' if the file is BeamFormed RAW data; ``TAB'' if the
 file is BeamFormed Processed data (will usually be ``TAB'').  \\
  \small{\verb|BF_VERSION|} & Attr. & \verb|string| & --- & BeamFormed data format version number.\\  
%  \small{\verb|NOF_STATIONS|} & Attr.  & \verb|int|& --- & Number of stations used
%    within this Beam \\
%  \small{\verb|STATIONS_LIST|}  & Attr. & \verb|array<string,1>| & --- & List of
%    stations used for this Beam (e.g.~``CS002'', ``RS106'', ``DE603''); list must match the NOF\_STATIONS
%    numerically \\
  \small{\verb|PRIMARY_POINTING_|} &  &  & & \\
  \small{\verb| DIAMETER|} & Attr. & \verb|double| & arcmin & FWHM of the
  sub-array pointing at zenith at center frequency. \\
  \small{\verb|POINT_RA|} & Attr. & \verb|double| & deg & J2000 right ascension of
    sub-array pointing at start of observation, in degrees (at LOFAR core).\\
  \small{\verb|POINT_DEC|} & Attr. & \verb|double| & deg & J2000 declination of station
    beam at start of observation, in degrees (at LOFAR core).\\
  \begin{scriptsize}\textit{POINT\_ALTITUDE}\end{scriptsize} & Attr. & \verb|array<string,1>| & deg & Altitude 
    of the pointing
    at start of observation, per stations list, in the same order. \\
  \begin{scriptsize}\textit{POINT\_AZIMUTH}\end{scriptsize} & Attr. & \verb|array<string,1>| & deg & Azimuth 
  	of the pointing
    at start of observation, per stations list, in the same order. \\
  \small{\verb|CLOCK_RATE|}   & Attr.   & \verb|double|& default: MHz & Clock rate, in units of CLOCK\_RATE\_UNIT (MHz) \\
  \small{\verb|CLOCK_RATE_UNIT|}  & Attr.   & \verb|string| & --- & Clock rate units \\
  \small{\verb|NOF_SAMPLES|} & Attr. & \verb|int| & --- & number of time samples. \\
  \small{\verb|SAMPLING_RATE|}   & Attr.  & \verb|double|  & default: MHz & Sampling rate, in units of SAMPLING\_RATE\_UNIT \\
  \small{\verb|SAMPLING_RATE_UNIT|}  & Attr.  & \verb|string| & --- & Sampling rate
    units \\
  \small{\verb|SAMPLING_TIME|}  & Attr.  & \verb|double|  & default: $\mu$s & Sampling time is
    1/SAMPLING\_RATE, in units of SAMPLING\_TIME\_UNIT \\
  \small{\verb|SAMPLING_TIME_UNIT|}  & Attr.  & \verb|string| & --- & Sampling time
    units \\
  \small{\verb|TOTAL_INTEGRATION_TIME|}  & Attr.  & \verb|double| & default: s & Total
    integration time = SAMPLING\_TIME * NOF\_SAMPLES \\
  \small{\verb|TOTAL_INTEGRATION_|} &  &  & & \\
  \small{\verb| TIME_UNIT|}  & Attr.  & \verb|string| &--- & Total
    integration time units \\
  \small{\verb|CHANNELS_PER_SUBBAND|}  & Attr. & \verb|int|& --- & Number of
    channels for each subband \\
  \small{\verb|SUBBAND_WIDTH|}   & Attr.  & \verb|double|  &  default: MHz &
  Subband width, usually
    0.156250 or 0.1953125 \\
  \small{\verb|SUBBAND_WIDTH_UNIT|}  & Attr.  & \verb|string| & --- & Subband width
    units \\
  \small{\verb|CHANNEL_WIDTH|}   & Attr.  & \verb|double|  &  default: MHz & Channel width is
    equal to the SUBBAND\_WIDTH / CHANNELS\_PER\_SUBBAND \\
  \small{\verb|CHANNEL_WIDTH_UNIT|}  & Attr.  & \verb|string| & --- & Channel
    width units \\
%  \small{\verb|TOTAL_INTEGRATION_TIME|} & Attr. & \verb|double| & s & Total integration time
%  of the observation. \\
  \small{\verb|TOTAL_BANDWIDTH|} & Attr. & \verb|double| & MHz & Total bandwidth (excluding gaps). \\
  \begin{scriptsize}\textit{WEATHER\_STATIONS\_LIST*} \end{scriptsize}& Attr. & \verb|array<string,1>| & --- & List of stations with weather information. \\
  \begin{scriptsize}\textit{WEATHER\_TEMPERATURE*}\end{scriptsize} & Attr. & \verb|array<double,1>| & $^\circ $C & Approximate
  outside temperature;  order must match the listing in WEATHER\_STATIONS\_LIST attribute.  (* See note below) \\
  \begin{scriptsize}\textit{WEATHER\_HUMIDITY*} \end{scriptsize}& Attr. & \verb|array<double,1>| & --- & Approximate humidity 
  (\%);  order must match the listing in WEATHER\_STATIONS\_LIST attribute (* See note below) \\
  \begin{scriptsize}\textit{SYSTEM\_TEMPERATURE*} \end{scriptsize}& Attr. &
  \verb|array<double,1>| & K & System temperature for the various 
  stations ;  order must match the listing in WEATHER\_STATIONS\_LIST attribute.  (* See note below) \\
%  \verb|SysLog| & Group & \verb|---| & --- &  Container for system-wide log object \\
  \hline
  \small{\verb|SYS_LOG|} & Group & --- & --- &   Container for system-wide log object \\
  \small{\verb|DYNSPEC_{NNN}|} & Group & --- & --- & Container for individual
    Dynamic Spectrum objects \\
  \small{\verb|TILED_DYNSPEC|} & Group & --- & --- & Container for tiled
    Dynamic Spectrum \\
  \hline
\end{supertabular}
\begin{small}
\begin{flushleft}
  *\textit{WEATHER\_TEMPERATURE, WEATHER\_HUMIDITY and SYSTEM\_TEMP are not available for all stations.}
\end{flushleft}
\end{small}
%\caption{Additional Root group attributes for LOFAR Dynamic Spectrum Data.}
%\label{tab:root attributes}
%\end{table}

Some of these keywords need further clarification:

\begin{itemize}
\item \verb|BF_FORMAT| \& \verb|BF_VERSION| -- These are required because the
dynamic spectrum pipelines used part of the BeamFormed software.
\end{itemize}
  
%%_______________________________________________________________________________
%% Subsection: The System Logs Group (SYS_LOG)

\input groups_sys_log

%%_______________________________________________________________________________
%%                                                             The DYNSPEC Group

\subsection{The Dynamic Spectrum Group ({\tt DYNSPEC\_\{NNN\}})}
\label{sec:image group}

The \texttt{DYNSPEC} group will be an HDF5 group serving as a container for the
four sub groups described below. A \texttt{DYNSPEC} group is designed to be as
complete and self-contained as possible, and will contain relevant data and 
metadata for a particular processed dynamic spectrum of a LOFAR observation. 
However, any breakout protocol will be required to inherit some or all ROOT group
attributes in order to function as a stand-alone dynamic spectrum.  
The adopted form allows for relatively simple extraction and
conversion in a FITS-compatible form.

\begin{figure}[htbp] 
  \begin{center}
     \includegraphics[scale=0.65]{figures/DynspecGroup.eps}
   \end{center}
\caption{DYNSPEC group structure.\label{fig:dynspec_group}}
\end{figure}

A \texttt{DYNSPEC} group (Figure \ref{fig:dynspec_group}) will comprise four sub groups.  These
four groups, two hierarchical levels below the ``ROOT group'' in a LOFAR Dynamic
Spectrum Data file (Figure \ref{fig:high-level structure}), will be 

\begin{itemize}
\item A \verb|COORDINATES| group that will contain one subgroup of each of the
	following kinds: 
	\verb|TIME_COORD|, \verb|SPECTRAL_COORD|, \verb|POLARIZATION_COORD|, 
	which describe various axes of the associated dataset.
\item A \verb|DATA| group that will contain a dataset array.
\item An \verb|EVENT| group that will be a tabular list of localized events
	(flares, bursts, etc.).
\item A \verb|PROCESS_HISTORY| group, which will be a meta-data container holding
	various processing products such as log files, parameter sets, RFI
	mitigation tables, etc.. 
\end{itemize}

%Figure XXX illustrates the form of an \texttt{DYNSPEC Group} in a 
%LOFAR Dynamic Spectrum Data file. Most of the relevent \texttt{DYNSPEC Group}
%metadata will be contained within the Coordinates groups (\textit{see
%\ref{sec:coordinates group}, ``Coordinates group''}) 

Table \ref{tab:Image group Attributes} 
lists the metadata stored within the \texttt{DYNSPEC} Group.

%\begin{table}[htbp]
%\centering
\tablefirsthead{%
	\hline
	\sc Field/Keyword & \sc H5Type & \sc Type & \sc Unit & \sc Description
  \\
	\hline \hline}
\tablehead{%
	\hline
	\sc Field/Keyword & \sc H5Type & \sc Type & \sc Unit & \sc Description
  \\
	\hline \hline}
\tabletail{%
	\hline
	\multicolumn{5}{|r|}{\small\sl continued on next page}\\
	\hline}
\tablelasttail{\hline}
\bottomcaption{\texttt{DYNSPEC Group} Attributes.
\label{tab:Image group Attributes}}
\begin{supertabular}{|p{3.5cm}p{1cm}p{2.5cm}p{3.5cm}p{5cm}|}
  %\hline
  %\sc Field/Keyword & \sc H5Type & \sc Type & \sc Unit & \sc Description \\
  %\hline \hline
  \small{\verb|GROUPTYPE|}              & Attr. & string & \verb|`DYNSPEC'| & \small{\texttt{LOFAR}} group type \\
    \hline
  \small{\verb|DYNSPEC_START_UTC|} & Attr. & \verb|string| & hh:mm:ss.sssssssss & Start time of obs (UTC time, 24-h clock). \\
  \scriptsize{\textit{DYNSPEC\_START\_MJD}} & Attr. & \verb|string| & \small{NNNNNN.NNNNNNN} & Start time of obs (MJD). \\
  \scriptsize{\textit{DYNSPEC\_START\_TAI}} & Attr. & \verb|string| & hh:mm:ss.sssssssss & Start time of obs (International Atomic Time). \\
  \small{\verb|DYNSPEC_STOP_UTC|} & Attr. & \verb|string| & hh:mm:ss.sssssssss & End time of obs (UTC time, 24-h clock). \\
  \scriptsize{\textit{DYNSPEC\_STOP\_MJD}}  & Attr. & \verb|string| & \small{NNNNNN.NNNNNNN} & End time of obs (MJD).  \\
  \scriptsize{\textit{DYNSPEC\_STOP\_TAI}} & Attr. & \verb|string| & hh:mm:ss.sssssssss & End time of obs (International Atomic Time). \\
  \small{\verb|DYNSPEC_BANDWIDTH|} & Attr. & \verb|double| & MHz & Bandwidth (excluding gaps). \\
  %\verb|TOTAL_INTEGRATION_TIME| & Attr. & \verb|double| & s & Total integration time of the observation. \\
  \small{\verb|BEAM_DIAMETER|} & Attr. & \verb|double| & arcmin & FWHM of the beams at zenith at center frequency. \\
  \small{\verb|TRACKING|} & Attr. & \verb|string| & -- & `J2000' = tracking ON; `LMN' = tracking OFF; TBD = tracking
    of solar system object \\
   % from "beam"
  \small{\verb|TARGET|} & Attr.   & \verb|array<string,1>| & --- & targets/sources observed within this Beam \\
  \small{\verb|ONOFF|} & Attr.   & \verb|string| & --- & e.g. `ON\_0' or `OFF\_0', `OFF\_1', ... \\
  \small{\verb|POINT_RA|} & Attr. & \verb|double| & deg & J2000 right ascension of the center of
  the beam at start of observation, in degrees (at LOFAR core). \\
  \small{\verb|POINT_DEC|} & Attr. & \verb|double| & deg &  J2000 declination of the center of the
  beam at start of observation, in degrees (at LOFAR core). \\
  \small{\verb|POSITION_OFFSET_RA|} & Attr. & \verb|double| & deg & RA offset of beam from sub-array pointing at start of observation (degrees). \\
  \small{\verb|POSITION_OFFSET_DEC|} & Attr. & \verb|double| & deg & Dec offset
  of beam from sub-array pointing at start of observation (degrees). \\
  \small{\verb|BEAM_DIAMETER_RA|} & Attr. & \verb|double| & arcmin & The diameter of the beam ellipse
  in RA. \\
  \small{\verb|BEAM_DIAMETER_DEC|} & Attr. & \verb|double| & arcmin & The diameter of the beam ellipse 
  in Dec. \\
   \small{\verb|BEAM_FREQUENCY_MAX|} & Attr. & \verb|double| & default: MHz & Beam maximum frequency \\
   \small{\verb|BEAM_FREQUENCY_MIN|} & Attr. & \verb|double| & default: MHz & Beam minimum frequency \\

  \small{\verb|BEAM_FREQUENCY_CENTER|} & Attr. & \verb|double| & default: MHz & Beam center
  				frequency: $0.5\cdot \left( \mbox{max}+ \mbox{min}\right)$. \\
  \small{\verb|BEAM_FREQUENCY_UNIT|} & Attr. & \verb|string| & --- & Units for \verb|BEAM_FREQUENCY_MAX|,
  \verb|BEAM_FREQUENCY_MIN| and \verb|BEAM_FREQUENCY_CENTER|. \\ 
  \small{\verb|BEAM_NOF_STATIONS|}    &    & \verb|int|& --- & N. of stations used for this beam \\
  \small{\verb|BEAM_STATIONS_LIST|}   &    & \verb|array<string,1>|& --- & List of stations used for this beam\\
  \small{\verb|DEDISPERSION|}   & Attr.   & \verb|string| & --- & Were the
  data dedispersed incoherently (\verb|INCOHERENT|), dedispersed coherently
  (\verb|COHERENT|), or (usually) not dedispersed (\verb|NONE|)? \\
  \small{\verb|DISPERSION_MEASURE|}   & Attr.   & \verb|double| & default: pc/cm$^{3}$ &  The dispersion 
  measure applied to the data, if data were de-dispersed \\
  \small{\verb|DISPERSION_MEASURE_UNIT|}   & Attr.   & \verb|string| & --- &
  The dispersion measure units. \\
  \small{\verb|BARYCENTER|}   & Attr. & \verb|bool| & --- & Are the data barycentered?
  (True/False)\\
  \small{\verb|STOKES_COMPONENTS|} & Attr. & \verb|array<string,1>| & --- & Stokes
  components for which data are attached to this group. \\
  \small{\verb|COMPLEX_VOLTAGE|} & Attr. & \verb|bool| & --- & Is the data in Complex
  Voltages (Xreal Ximg, Yreal, Yimg)? (True/False) \\
  \small{\verb|SIGNAL_SUM|} & Attr. & \verb|string| & --- & Was the signal summed coherently
  (\texttt{COHERENT}) or incoherently (\texttt{INCOHERENT\_SKY}, \texttt{INCOHERENT\_PRIMARY\_POINTING})? \\
  \hline
  \small{\verb|COORDINATES|} & Group & --- & --- &  container for Coordinates
  Group for this Dynamic Spectrum \\
  \small{\verb|DATA|} & Group & --- & --- &  container for dataset for this
  Dynamic Spectrum \\
  \small{\verb|EVENT|} & Group & --- & --- & Container for Event list for this
  Dynamic Spectrum  \\
  \small{\verb|PROCESS_HISTORY|} & Group & --- & --- & Container for Processing History for this
  Dynamic Spectrum \\
  \hline
\end{supertabular}
%\caption{\texttt{DynSpec Group} Attributes.}
%\label{tab:Image group Attributes}
%\end{table}

Some of the keywords in Table \ref{tab:Image group Attributes} need further clarification:

\begin{itemize}
\item \verb|POINT_RA| \& \verb|POINT_DEC| -- J2000 right ascension and 
  declination of the center of the beam at start of observation, in degrees (at the LOFAR core,
  as the positions will be slightly different for solar system objects).
\item \verb|STOKES_COMPONENTS| -- Providing both flexibility and readability,
  the Stokes components for which data are attached to the Beam Group, 
  are stored as \verb|array<string,1>|. 
  The attribute has the flexibility to host different combinations and types of
  Stokes data, such as e.g. 
  \begin{verse}
    \verb|STOKES_COMPONENTS| = [``I''] \\
    \verb|STOKES_COMPONENTS| = [``I'',``Q'',``U'',``V''] \\
    \verb|STOKES_COMPONENTS| = [``I'',``Q''] \\
    \verb|STOKES_COMPONENTS| = [``L'',``R''] \\
    \verb|STOKES_COMPONENTS| = [``XX'',``XY'',``YX'',``YY''].
  \end{verse}
\item \verb|ON_OFF|: Specify whether the beam was ``ON'' source, or ``OFF''
   source. Also, specify the number of the beam (e.g. ``OFF\_0'', 
   ``OFF\_1'', ...). Same for ``ON'' beams: In most cases, only one 
   ``ON'' beam will be used (``ON\_0''), but for an extended source, multiple
   ``ON'' beams are possible.  
\end{itemize}

%%____________________________________________________________________

\subsection{The Tiled Dynamic Spectrum Group ({\tt TILED\_DYNSPEC})}
\label{sec:tiled dynspec group}

This group has the same structure as the \texttt{DYNSPEC groups}. In the case
where different stations observe at different frequencies, this group can be used
to accomodate a composite dynamic spectrum, combining the frequency information of
different \texttt{DYNSPEC groups}. A LOFAR Dynamic Spectrum Data file has
either zero or one \texttt{TILED\_DYNSPEC groups}.

\begin{comment}
  The final format is not yet defined. This group should include
  information from which DynSpec groups the information as taken.
\end{comment}

%%_______________________________________________________________________________
%%                                            The Coordinates Group (COORDINATES)

\subsection{The Coordinates Group ({\tt COORDINATES})}
\label{sec:coordinates group}

The Coordinates Group acts as a container to take up a collection of coordinates,
as described in the subsequent sections below. Besides this function as a
container -- grouping together embedded coordinate objects -- the
Coordinates Group also provides basic reference frame information,
which is required for the proper transformation of quantities to other
reference systems.

%% ------
\input coordinates_group_table

The attributes, as presented in Table \ref{tab:coordinates-group},  summarize the
overall characteristics of the set of coordinates collected within a
\texttt{COORDINATES Group}. Specifially, for a LOFAR \texttt{DYNSPEC Group},
the following values are used:
\begin{itemize}
\item \verb|NOF_COORDINATES| -- The number of coordinate objects/groups
contained within the coordinates group. For Dynamic Spectrum Data, this is
usually 3
(\texttt{TIME\_COORD, SPECTRAL\_COORD, POLARIZATION\_COORD}).
\item \verb|NOF_AXES| -- The number of coordinate axes associated with the 
  coordinate objects. For Dynamic Spectrum Data, the number of axes will usually
  be 3, i.e.~1 for each of the subgroups 	
  \texttt{TIME\_COORD, SPECTRAL\_COORD, POLARIZATION\_COORD}.
% 	\texttt{direction, linear, tabular, spectral, stokes}.
\end{itemize}

%% ------
\input coordinates_frames_location
  
The different axes stored within this container are described in the remainder of this
section. 
More information on coordinates in LOFAR data files in general can be found 
in ICD-002 \cite{lofar.icd.002}. 

%%____________________________________________________________________
%%                                                     Time coordinate

\subsubsection{Time coordinate}
\label{sec:coord-linear}

This group describes the time coordinate of the associated 
\texttt{DYNSPEC Group}. The associated metadata are given in Table
\ref{tab:time coords attributes}.

\begin{table}
\begin{center}
  \begin{tabular}{|lllrp{5cm}|}
    \hline
    \sc Field/Keyword & \sc H5Type & \sc Type & \sc Value & \sc Description \\
    \hline \hline
    \small{\verb|GROUPTYPE|} & Attr. & \verb|string| & \small{\texttt{`TimeCoord'}} & Group Type descriptor \\
    \small{\verb|COORDINATE_TYPE|} & Attr. & \verb|string| & \small{\texttt{`Time'}} & Coordinate Type descriptor \\
    \small{\verb|STORAGE_TYPE|} & Attr. & \small\verb|array<string,1>| & 
    	\small \verb|`Linear'| $|$ \verb|`Tabular'| & coordinate storage type \\ 
    \small{\verb|NOF_AXES|} & Attr. & \verb|int| & 1 & N of coordinate axes \\
    \small{\verb|AXIS_NAMES|} & Attr. & \verb|array<string,1>| & [`Time'] & World axis names \\
    \small{\verb|AXIS_UNITS|} & Attr. & \verb|array<string,1>| & [`s'] & World axis units \\
    \small{\verb|REFERENCE_VALUE|} & Attr. & \verb|array<double,1>| & --- & Reference value \\
    \small{\verb|REFERENCE_PIXEL|} & Attr. & \verb|array<double,1>| & --- & Reference pixel \\
    \small{\verb|INCREMENT|} & Attr. & \verb|array<double,1>| & --- & Coordinate increment \\
    \small{\verb|PC|} & Attr. & \verb|array<double,1>| & ---& The World Coordinate Reference scaling delta matrix \\
    \small{\verb|AXIS_VALUES_PIXEL|} & Attr. & \verb|array<double,1>| & --- & Reference pixels \\
    \small{\verb|AXIS_VALUES_WORLD|} & Attr. & \verb|array<double,1>| & --- & Reference values \\
    \hline
  \end{tabular}
  \caption{Time Coordinate Attributes\label{tab:time coords attributes}}
\end{center}
\end{table}

Specifially, for a LOFAR \texttt{DYNSPEC Group}, the following values are used:
\begin{itemize}
\item \verb|COORDINATE_TYPE| --- 
	The coordinate type = `Time'.
\item \verb|STORAGE_TYPE| ---
  The storage type can be either `Linear' or `Tabular'. For dynamic spectra, 
  it will be `Linear' in most cases.
\item \verb|NOF_AXES| --- 
	The number of coordinate axes =  1.
\item \verb|AXIS_NAMES| --- 
	The names of the axis = `Time'.
\item \verb|AXIS_UNITS| --- 
	The units of the coordinate axis, e.g. = `microseconds'.
\item \verb|REFERENCE_VALUE| ---
	The World Coordinate Reference value (CRVAL) will be the 0,0 location of the dataset (start time of each sample).
\item \verb|REFERENCE_PIXEL| ---
	The World Coordinate Reference pixel (CRPIX) will be the 0,0 location of the dataset.
\item \verb|INCREMENT| ---
	The World Coordinate increment (CDELT) is the time bin (SAMPLING\_TIME).
\item \verb|PC| ---
	The World Coordinate Reference scaling delta matrix is flat (1, 0) for Dynamic Spectrum data.

\end{itemize}

More detail on time coordinates for LOFAR data files can be found in 
Section 4.3.1 of ICD-002 \cite{lofar.icd.002}. 

%%____________________________________________________________________
%%                                                 Spectral coordinate

\subsubsection{Spectral coordinate}
\label{sec:spectral coordinate}

This group describes the spectral coordinate of the associated 
\texttt{DYNSPEC Group}. The associated metadata are given in Table
\ref{tab:spectral coords attributes}.

\vspace{20pt}
\begin{table}[htbp]
\begin{small}
\begin{center}
  \begin{tabular}{|llllp{5cm}|}
    \hline
    \sc Field/Keyword & \sc H5Type & \sc Type & \sc Value & \sc Description \\
    \hline \hline
    \verb|GROUPTYPE|  & Attr.  & \verb|string| & `SpectralCoord' & Group type descriptor\\
    \verb|COORDINATE_TYPE| & Attr. & \verb|string| & `Spectral' & Coordinate Type  descriptor \\
    \verb|STORAGE_TYPE| & Attr. & \small\verb|array<string,1>| & \small
    	\verb|`Linear'| $|$ \verb|`Tabular'| & coordinate storage type \\    
    \verb|NOF_AXES|   & Attr. & \verb|int| & 1 & nof. coordinate axes \\
    \verb|AXIS_NAMES| & Attr. & \verb|array<string,1>| & [`Frequency'] & World axis names \\
    \verb|AXIS_UNITS| & Attr. & \verb|array<string,1>| & [`Hz'] & World axis units \\
    \verb|REFERENCE_VALUE| & Attr. & \verb|array<double,1>| & --- & Reference value (CRVAL) \\
    \verb|REFERENCE_PIXEL| & Attr. & \verb|array<double,1>| & --- & Reference pixel (CRPIX) \\
    \verb|INCREMENT| & Attr. & \verb|array<double,1>| & --- & Coordinate increment (CDELT) \\
    \verb|PC|    & Attr. & \verb|array<double,1>| & ---& The World Coordinate Reference scaling delta matrix \\
    \verb|AXIS_VALUES_PIXEL| & Attr. & \verb|array<double,1>| & --- & Reference pixels \\
    \verb|AXIS_VALUES_WORLD| & Attr. & \verb|array<double,1>| & --- & Reference values \\
 \hline
\end{tabular}
\end{center}
\end{small}
\caption{Spectral Coordinate Attributes}
\label{tab:spectral coords attributes}
\end{table}

Specifially, for a LOFAR \texttt{DYNSPEC Group}, the following values are used:

\begin{itemize}
  \item \verb|COORDINATE_TYPE| ---
  The coordinate type = `Spectral'.
  \item \verb|STORAGE_TYPE| ---
  The storage type can be either `Linear' or `Tabular'. For dynamic spectra, 
  it will be `Tabular' in most cases\footnote{ Given the flexibility concerning the arrangement of frequency 
channels or subbands, the values along this coordinate axis might be linear, but do
not necessarily have to be.}.
  \item \verb|NOF_AXES| ---
  The number of coordinate axes = 1.
  \item \verb|AXIS_NAMES| ---
  The names of the axis = `Frequency'.
  \item \verb|AXIS_UNITS| ---
  The units of the coordinate axis is = `MHz'.
  \item \verb|REFERENCE_VALUE| ---
  The World Coordinate Reference value (CRVAL) will be the 0,0 location of the dataset.
  \item \verb|REFERENCE_PIXEL| ---
  The World Coordinate Reference pixel (CRPIX) will be the 0,0 location of the dataset.
  \item \verb|INCREMENT| ---
  The World Coordinate increment (CDELT) is the frequency bin (CHANNEL\_WIDTH).
  \item \verb|PC| ---
  The World Coordinate Reference scaling delta matrix is flat (1, 0) for Dynamic Spectrum data.
  \item \verb|AXIS_VALUES_PIXEL| ---
  Reference pixels - List of the subbands. (See Sec.~\ref{sec:coordinate examples}).
  \item \verb|AXIS_VALUES_WORLD| ---
  Reference values -- List of the equivalent frequencies of the list
  of subbands. (See Sec.~\ref{sec:coordinate examples})
\end{itemize}

More detail on spectral coordinates for LOFAR data files can be found in 
Section 4.3.2 of ICD-002 \cite{lofar.icd.002}. 

%%____________________________________________________________________
%%                                             Polarization coordinate

\subsubsection{Polarization coordinate}
\label{sec:stokes coordinate}

This group describes the polarization contents of the associated 
\texttt{DYNSPEC Group}.
For Dynamic Spectrum data, only one axis of this type will be used, i.e. 
\verb|NOF_AXES|=1.

Within each \texttt{DYNSPEC Group}, there are either one or four Stokes Groups.
If the data are summed, then there is only the Stokes I information. If the data
are not
summed, then there are four Stokes tables (I, Q, U, V or XX, XY, YX, YY), one 
per polarization.
Table~\ref{tab:stokes attributes} lists the Attributes in the Stokes Groups. 

\begin{center}
\begin{table}
  \begin{tabular}{|lllrp{5.5cm}|}
    \hline
    \sc Field/Keyword & \sc H5Type & \sc Type & \sc Value & \sc Description \\
    \hline \hline
   \small{\verb|GROUP_TYPE|} & Attr. & \verb|string| &
   {\texttt{`PolarizationCoord'}} & Coordinate Type descriptor \\
   \small{\verb|COORDINATE_TYPE|} & Attr. & \verb|string| & {\texttt{`Polarization'}} & Coordinate Type descriptor \\
     \small{\verb|STORAGE_TYPE|} & Attr. & \small\verb|array<string,1>| & \small
    	\verb|`Tabular'| & coordinate storage type \\    
    \small{\verb|NOF_AXES|}              & Attr. & \verb|int| & 1 & N of coordinate axes \\
    \small{\verb|AXIS_NAMES|}          & Attr. & \verb|array<string,1>| & `Polarization' & World axis names \\
    \small{\verb|AXIS_UNITS|}          & Attr. & \verb|array<string,1>| & --- & World axis units \\
    \hline
  \end{tabular}
  \caption{Stokes Groups Attributes}
  \label{tab:stokes attributes}
  \end{table}
\end{center}

Specifially, for a LOFAR \texttt{DYNSPEC Group}, the following values are used:

\begin{itemize}
  \item \verb|COORDINATE_TYPE| --- 
  The coordinate type = `PolarizationCoord'.
  \item \verb|STORAGE_TYPE| --- The descriptor for the underlying storage
  type for this coordinate, of value \verb|`Tabular'|.
  \item \verb|NOF_AXES| --- 
  The number of coordinate axes = 1.
  \item \verb|AXIS_NAMES| --- 
  The names of the axis is = `Polarization'.
  \item \verb|AXIS_UNITS| --- 
  The units of the coordinate axis is = `NONE'.
\end{itemize}

More detail on polarization coordinates for LOFAR data files can be found in 
Section 4.3.3 of ICD-002 \cite{lofar.icd.002}. 

%%____________________________________________________________________
%%                                        The Dynamic Spectrum Dataset

\subsection{The Dynamic Spectrum Dataset}
\label{sec:dynspec dataset}

\begin{table}[htbp]
  \centering
  \begin{tabular}{|llrp{6cm}|}
    \hline
    \sc Field/Keyword & \sc Type & \sc Value & \sc Description \\
    \hline \hline
    \small\verb|GROUPTYPE| & \texttt{string} &\verb| `Data'|& Group type descriptor \\
    \small\verb|WCSINFO| & \texttt{string} &\verb| `/Coordinates'| & Path to the coordinates group describing the transformation for array pixel axes to world coordinates. \\
    \small{\verb|DATASET_NOF_AXES|} & \verb|int| & N & Number of array axes of the dataset \\
    \small{\verb|DATASET_SHAPE|} & \verb|array<int,1>| & --- &
    Shape of the dynamic spectrum data array. \\
    \hline
    \small{\verb|DATA|} & \verb|array<double,N>| &  & \\
    \hline
  \end{tabular}
  \caption{Attributes attached to the dynamic spectrum dataset array.}
  \label{tab:image dataset}
\end{table}


A dynamic spectrum \texttt{DATA} group will most often be a subgroup
of a \texttt{DYNSPEC Group} container and consist of an HDF5
``dataset,'' which, as defined in the HDF5 documentation
\cite{hdf5,hdf5.ug}, is ``stored in two parts: a header and a data
array.''  However, with the adoption of a so-called ``Coordinates
group,'' which contains all the relevent information, scale and unit
metadata, \texttt{Data} group attributes will be limited.

The dataset array will (usually) be a 3-D data structure.
The usual dimensionality of a
\texttt{Data} group's dataset will be 3 (\texttt{N\_AXIS=3}), wherein the data
cube (or cubes) will be defined in (C-type order) \textit{Polarization},
\textit{Spectral}, \textit{Time}.

LOFAR Dynamic Spectrum Data files will limit attributes to nominal keyword-value
pairs as much as possible, with a thought toward potential future user requests
for FITS format images (e.g.~one dynamic spectrum per FITS file).  
See \S\ \ref{sec:coordinates group} ``The Coordinates
group,'' for a detailed specification of \texttt{Data} group header attributes. 

The Subband/Channel information will be stored as either 1-D or N-D tables or
arrays.  This is where the bulk of the data reside.  The
general structure is such that channels are columns and time bins are rows.  If
using 1-D tables/arrays, then each subband will be in its own table/array.  If
using N-D tables/arrays, then each subband will be a plane in a data cube.  
Time increments are quantized and are filled (with NaN values)
for gaps;  the time axis is usually linear.  For each subband, the channels are
quantized and filled for gaps, so within each subband table/array, the frequency
axis is linear.  However, since the subbands can have large gaps, the
overall frequency axis is not linear.  The frequency coordinate axis information
is stored in the Spectral Coordinate Group in the Dynamic Spectrum Data file.   

For example, Subband 0 can start at 140 MHz, Subband 1 at 150 MHz and Subband 2 at
165 MHz;  each Subband has the same number of channels (subdivisions), say 512,
all with the same channel widths, and same number of time increments (rows), say
2000000.  Therefore, there is a gap between Subbands 1 and 2 in this example.  One
cannot assume that the last channel in Subband (N-1) is followed directly in
frequency by the first channel in the next Subband N.  Gaps are accounted for in
the Spectral Coordinate Group. 

%% --------------------------

There are several possibilities of sub-observing modes. Five
sub-modes are listed below, but please note that this is not yet a full set of
sub-modes. 

\begin{enumerate}

\item \textsl{Raw data}

The default data stored in each subband table is full resolution, polarized voltages
from the dipoles. These voltages are stored as 16-bit complex pairs for both X and
Y. In other words, each sample is stored as (X-real, X-imaginary)(Y-real,
Y-imaginary) for a total of 64-bits (2 32-bit complex numbers).  This is the format
of the BF RAW data. 

\item \textsl{Stokes I with IncoherentSum}

  \begin{verse} \small
    \verb|STOKES_COMPONENTS| = [``I''] \\
    \verb|SIGNAL_SUM| = \verb|INCOHERENT|
  \end{verse}

  Data taken for \textbf{Stokes I} TiedArray observing mode with Incoherent
  Summing. Data are being stored as one number, the total intensity per unit
  time. Currently this number is being stored as a single float and will
  eventually be stored as a 16-bit integer.

\item{Stokes I with CoherentSum}
  
  \begin{verse} \small
    \verb|STOKES_COMPONENTS| = [``I''] \\
    \verb|SIGNAL_SUM| = \verb|COHERENT|
  \end{verse}

  Data taken for \textbf{Stokes I} TiedArray observing mode with Coherent
  Summing. Data are being stored as one number, the total intensity per unit
  time.  Currently this number is being stored as a single float and will
  eventually be stored as a 16-bit integer.  
  
\item{Full Stokes with IncoherentSum}
  
  \begin{verse} \small
    \verb|STOKES_COMPONENTS| = [``I'',``Q'',``U'',``V''] \\
    \verb|SIGNAL_SUM| = \verb|INCOHERENT|
  \end{verse}

  Data taken for \textbf{Full Stokes} TiedArray observing mode with Incoherent
  Summing. Data are being stored as 4 numbers, per unit time.  Currently these
  4 numbers are being stored as 4 floats and will eventually be stored as 4
  16-bit integers.
  
\item{Full Stokes with CoherentSum}
  
  \begin{verse} \small
    \verb|STOKES_COMPONENTS| = [``I'',``Q'',``U'',``V''] \\
    \verb|SIGNAL_SUM| = \verb|COHERENT|
  \end{verse}

  Data taken for \textbf{Full Stokes} TiedArray observing mode with Coherent
  Summing. It is stored as 4 numbers, per unit time.  Currently these 4 numbers
  are being stored as 4 floats and will eventually be stored as 4 16-bit
  integers.  
  
\end{enumerate}

%% --------------------------

\subsection{The Event Group ({\tt EVENT})}
\label{sec:event group}

The \texttt{Event} group in a Dynamic Spectrum Data file will be a table of
events detected within the dynamic spectrum, including their associated 
parameters.
The \texttt{Event} group header will
specify the fields (columns) of the table, and the number of events in the table
(rows). The attributes of this group are given in Table \ref{tab:attributes
event group}. 

\begin{comment}
  The precise format is not yet defined.  This should also include
  information on how the events were detected, i.e. the settings used
  for event trigging.
\end{comment}

\begin{table}[ht]
  \centering
  \begin{tabular}{|lllrl|}
    \hline
    \sc Field/Keyword &  \sc H5Type   & \sc Type      & \sc Value & \sc Description\\
    \hline \hline                    
    \small{\verb| GROUPTYPE|} & Attr. & string & \verb|`Event'| & Dynamic Spectrum group type \\
    \small{\verb| DATASET|}   & Attr. & string & \verb|`Event List'|   &  \\
    %\small{\verb| N_AXIS|}    & Attr. & int    & \verb|2|        & Number of data axes \\
    %\small{\verb| N_AXIS_1|}  & Attr. & string & \verb|`Fields'| & Axis of the data fields \\
    %\small{\verb| N_AXIS_2|}  & Attr. & string & \verb|`Event'|  & Axis of the event rows. \\
    \small{\verb| N_SOURCE|}  & Attr. & int    &                 & Number of data rows \\
    \small{\verb| FIELD_1|}   & Attr. & \verb|array<double,1>| &               & time \\
    \small{\verb| FIELD_2|}   & Attr. & \verb|array<double,1>| &               & frequency \\
    \small{\verb| FIELD_3|}   & Attr. & \verb|array<double,1>| &               & Peak Flux \\
    \small{\verb| FIELD_4|}   & Attr. & \verb|array<double,1>| &               & Integrated Flux \\
    \small{\verb| FIELD_5|}   & Attr. & \verb|array<double,1>| &               & duration \\
    \small{\verb| FIELD_6|}   & Attr. & \verb|array<double,1>| &               & $f_{max}-f_{min}$ \\
%    \small{\verb| FIELD7|}   & Attr. & double &               & Position angle \\
    \hline
  \end{tabular}
  \caption{Attributes of an Event group.}
  \label{tab:attributes event group}
\end{table}

%%_______________________________________________________________________________
%% Subsection: The Processing History Group (PROCESS_HIST)

\subsection{The Processing History Group ({\tt PROCESS\_HIST})}
\label{sec:processing group}

The data definition for the \texttt{PROCESS\_HISTORY} group 
is necessarily loose, and will accommodate a variety of ancillary
meta-data related to or produced by the various LOFAR processing
pipelines. Products such as DPPP log files, processing parameters
sets, RFI mitigation tables, etc. In fact, and due to the wide-ranging data types
and free-form ASCII format of many log files that the
\texttt{PROCESS\_HISTORY} group may encompass, this group will be a
catch-all envelop encapsulating information about all steps of
processing should the user need such information. And it is because
of this free-form nature of the meta-data that it is very difficult to
define a header describing attached data when it is not yet known just
what those data may include. An attempt has been made to provide by
example how this will or should appear in the \texttt{PROCESS\_HISTORY}
group header. The attributes of this group are given in Table
\ref{tab:processing history 
group}, and an example of the structure of this is group is shown in Figure \ref{fig:processing history group}.

\begin{table}[htbp]
  \centering
  \begin{tabular}{|llrp{7cm}|}
    \hline 
    \sc Field/Keyword  & \sc Type     & \sc Value & \sc Description \\
    \hline \hline 
     \small\verb|GROUPTYPE|        & \verb|string| &
     \small\verb|`ProcessHistory'| & \small{\verb|LOFAR|} group type \\
     \small\verb|OBSERVATION_PARSET|        & \verb|bool| &  & Observing parset present? (True/False) \\
     \begin{scriptsize}\textit{OBSERVATION\_LOG}\end{scriptsize}     & \verb|bool|      &            & Observing log present? (True/False) \\
     \begin{scriptsize}\textit{DYNSPEC\_PARSET}\end{scriptsize}      & \verb|bool|  &  & Dynamic Spectrum parset present? 
     									(True/False) \\
     \begin{scriptsize}\textit{DYNSPEC\_LOG}\end{scriptsize} & \verb|bool|  &  & Dynamic Spectrum log present? 
     									(True/False) \\
    \hline
  \end{tabular}
  \caption{Attributes of a PROCESS\_HISTORY group.}
  \label{tab:processing history group}
\end{table}

\begin{comment}
  The Figure has to be adapted to DYNSPEC case.
\end{comment}

\begin{figure}[htbp]
  \begin{center}
    \includegraphics{figures/ProcessingGroup2.eps}
 \caption{The PROCESS\_HISTORY group, nested tabulation  \label{fig:processing history group}}
 \end{center}

\end{figure}

As with all other Dynamic Spectrum Data file HDF5 groups and subgroups, the
\texttt{PROCESS\_HISTORY} group will be an HDF5 group, as a subgroup of a
\texttt{DYNSPEC Group}. 
The attributes will contain a brief summary of the appended processing files
contained therein, with pointers to tables containing the logging data, parameter
sets, etc..

\begin{comment}
  Include information on rebinning, etc.
\end{comment}

%% ------------------------------------------------------------------------------
%
%\section{Interfaces}
%\label{sec:interfaces}
%
%---/---
%
%\subsection{Interface requirements}
%
%---/---
%
%\subsection{Relation to other workpackages}
%
%---/---
%
%% ==============================================================================
%%
%%  Appendix
%%
%% ==============================================================================

\clearpage
\appendix

%%_______________________________________________________________________________
%% Discussion & open questions

\section{Discussion \& open questions}
\label{sec:discussion}

%% -------------------------------------------------------------------------------
\subsection{Dynamic Spectrum Data flow}

The dynamic spectrum data flow is shown in Figure \ref{fig:dataflow}.

\begin{figure}[htbp]
  \centering
  \includegraphics[scale=0.6,angle=90]{figures/HDF5-writer-DS-data-detail.ps}
  \caption{Dynamic Spectrum Data Flow}
  \label{fig:dataflow}
\end{figure}

%% -------------------------------------------------------------------------------

\subsection{Summary of HBA JOINED Discussion}
\label{sec:Summary of HBA JOINED Discussion}

\input hba_joined_appendix
%%_______________________________________________________________________________
%% LOFAR Filename Convention

\section{LOFAR Filename Convention}
\label{sec:filenames}

The LOFAR file naming convention is described in the document, \texttt{LOFAR-USG-ICD-005} \cite{lofar.icd.005}. Readers are encouraged to consult that document for specifics on LOFAR file naming conventions.

%%_______________________________________________________________________________
%% Coordinates group examples

\section{Coordinates Group examples}
\label{sec:coordinate examples}

An in-depth description -- including a number of examples -- can be
found in \texttt{LOFAR-USG-ICD-002} \cite{lofar.icd.002}. Readers are
encouraged to consult that document for specifics on the storage of
world coordinates information.

\newpage

\section*{\glossaryname}
\label{sec:glossary}
\addcontentsline{toc}{section}{\glossaryname}

\input lofar_common_glossary

\bibliographystyle{abbrv}
\bibliography{references}


\end{document}
