\svnInfo $Id$

\section{LOFAR Filename Convention}
\label{sec:filenames}

LOFAR data products will have file names of the form,

\begin{center}
  \verb|L<Observation ID>_<Optional Descriptors>_<Filetype>.<Extension>|
\end{center}

In this the individual fields are defined as follows:

\begin{enumerate}
\item \textbf{Observation ID} -- unique identifier of the observation as part of which (or based on which) this data product was created.
\item \textbf{Optional Descriptors} are used to further indicate the
  nature of the data stored within the file:
  \begin{verse}
    \begin{tabular}{|lll|}
      \hline
      \textbf{Description} & \textbf{Format} & \textbf{Example} \\
      \hline \hline
      Beam number (B)   & 2 digits   & \verb|Bxx|   \\
      Subband (SB)      & 3 digits   & \verb|SBxxx| \\
      Pencil Beam (PB)  & 3 digits   & \verb|PBxxx| \\
      Date (D)          & 8 digits   & \verb|D<yyyymmdd>| \\
      Date (D) \& Time  & 8+6 digits & \verb|D<yyyymmdd>T<hhmmss>| \\
      \hline
    \end{tabular}
  \end{verse}
  While the descriptors can be used to e.g. indicate a specific sub-band and/or
  beam, it is \underline{not to be used for ranges}.

\item \textbf{Filetype} is a marker for the contents of the
  file. There will be several different kinds of data produced by
  LOFAR, see table below. Importantly, filetype signifies the kind of
  LOFAR data that comprise the particular data file, and therefore,
  will also signal the appropriate interface control document for
  further reference, should users find that necessary. The options for
  the file type along with their abbreviations are listed in the table below.
  \input lofar_common_filetype
  
\item \textbf{Extension}
  \begin{verse}
    \begin{tabular}{|ll|}
      \hline
      \textbf{Extension} & \textbf{Type of data} \\
      \hline \hline
      \verb|.MS|     & CASA/casacore MeasurementSet \\
      \verb|.h5|     & HDF5 file \\
      \verb|.fits|   & FITS file \\
      \verb|.log|    & Logfile \\
      \verb|.parset| & A parset file \\
      \verb|.lsm|    & Local sky model \\
      \verb|.IM|     & CASA/casacore image file (PagedImage) \\
      \verb|.PD|     & ParmDB file generated by BBS \\
      \verb|.vds|    & Dataset description file \\
      \verb|.gds|    & Dataset description file \\
      \verb|.conf|   & Configuration file (mostly local to station) \\
      \hline
    \end{tabular}
  \end{verse}
  Files generated by CASA/casacore will continue the currently
  existing conventions using upper-case suffixes.
\end{enumerate}
