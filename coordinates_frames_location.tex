\svnInfo $Id$

%% ==============================================================================
%%
%% Table with the recognized values for the reference frame to specify a location
%%
%% ==============================================================================

\begin{table}[ht]
  \centering
  \begin{tabular}{|p{3.5cm}p{6cm}p{5cm}|}
    \hline
    \textsc{Reference~Position} & \textsc{Description} & \textsc{Comments} \\
    \hline \hline
    \verb|GEOCENTER|    & Center of the Earth. & \\
    \verb|BARYCENTER|   & Center of the solar system barycenter. &  \\
    \verb|HELIOCENTER|  & Center of the Sun. &  \\
    \verb|TOPOCENTER|   & ``Local''; in most cases this will mean:
    the location of the telescope. &  \\
    \verb|LSRK|         & Kinematic Local Standard of Rest: 20 km
    s$^{-1}$ in the direction of \verb|GALACTIC_II| $(56,+23)$. & Only to be
    used for redshifts and Doppler velocities, and spectral coordinate. \\
    \verb|LSRD|         & Dynamic Local Standard of Rest: 16.6 km
    s$^{-1}$ in the direction of \verb|GALACTIC_II| $(53,+25)$. &  \\
    \verb|GALACTIC|     & Center of the Galaxy: 220 km s$^{-1}$ in
    the direction of \verb|GALACTIC_II| $(90,0)$ w.r.t. \texttt{LSRD}. &  \\
    \verb|LOCAL_GROUP|  & Center of the Local Group: 300 km s$^{-1}$ in
    the direction of \verb|GALACTIC_II| $(90,0)$ w.r.t. \texttt{BARYCENTER}. &  \\
    \verb|RELOCATABLE|  & Relocatable center; for simulations. & Only to be used
    for spatial coordinates. \\
    \hline
  \end{tabular}
  \caption[Recognized values for the coordiante reference frame]{Recognized values for the reference frame to specify a
    location; values and descriptions have been adopted from the
    ``Space-Time Coordinate Metadata for the Virtual Observatory''
    \cite{ivoa.stc}, as produced by the IVOA Data Model Working Group.}
  \label{tab:reference frames location}
\end{table}
