\svnInfo $Id$

\paragraph{Standard Data Types.}

The following table describes the short-form name for each type used
throughout the rest of this document, it's logical meaning in the context of
the astronomical data product, and the physical storage which must be
allocated to it within the HDF5 data model.  Future versions of this document
may augment these types but will not remove support for existing types.

\begin{center}
  \begin{tabular}{rlp{7.8cm}}
    \textsc{Name}        & \textsc{Logical Type}                  & \textsc{Physical Storage Allocation} \\
    \hline
    \verb|short|         & Integer; range $-2^{15}$ to $2^{15}-1$ & 16 bit signed two's complement integer \\
    \verb|int|           & Integer; range $-2^{31}$ to $2^{31}-1$ & 32 bit signed two's complement integer \\
    \verb|unsigned|      & Integer; range $0$ to $2^{32}-1$       & 32 bit unsigned integer \\
    \verb|unsigned long long| & Integer; range $0$ to $2^{64}-1$  & 64 bit unsigned integer \\
    \verb|float|         & Single precision floating-point        & IEEE 754-2008 \cite{ieee.754-2008} ``binary32'' floating point (1 bit sign, 8 bit exponent, 23 bit mantissa) \\
    \verb|double|        & Double precision floating-point        & IEEE 754-2008 ``binary64'' floating point (1 bit sign, 11 bit exponent, 52 bit mantissa)\\
    \verb|complex<type>| & Complex form of \verb|type|            & Compound type; the part at the lower memory location is real \\
    \verb|bool|          & Boolean true/false                     & 32 bit signed two's complement integer; non-zero denotes ``true''\\
    \verb|string|        & Text                                   & Null-terminated string of 8 bit bytes. The lower 128 values interpreted as ASCII encoded characters\\
    \verb|array<type,N>| & Array of \verb|type| with rank \texttt{N} & \\
    \hline
  \end{tabular}
\end{center}

Note that data may be written with either ``big-endian'' or ``little-endian''
byte ordering; either is valid within the context of this document.

